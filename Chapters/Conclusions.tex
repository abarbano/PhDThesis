\chapter{Conclusions}
\label{chap:conclusions}
This thesis presented the measurements of $\Dsplus$ meson production in pp, p-Pb and Pb-Pb collisions.
The analysis was performed via the reconstruction of the displaced-vertex decay topology $\DstoKKpi$ at 
mid-rapidity with the ALICE detector. \\

In comparison to previous ALICE publications based on the same data sample of pp collisions at $\s=7$ TeV, the present
results have total uncertainties reduced by a factor of about two thanks to improvement
in the reconstruction and detector alignments and to several optimisation in the analysis procedure. The $\Dsplus$ $\pt$-differential cross 
section was measured in the transverse-momentum interval $2<\pt<12 \, \Gevc$ and was found to 
be well described by perturbative QCD calculations.\\


The $\Dsplus$ nuclear modification factor $\RAA$ was measured in central, semi-peripheral and peripheral Pb-Pb collisions at 
$\sNN=5.02$ TeV in the $\pt$ range $4< \pt <16 \, \Gevc$. This observable is sensitive to the interactions of charm quarks 
with the Quark-Gluon Plasma medium formed in the collision. The obtained results indicate a
substantial modification of the $\Dsplus$ meson $\pt$ distribution as compared to pp collisions
with a maximum suppression of the $\Ds$ yield at $\pt \sim 6-10\, \Gevc$, for the 10\% most central Pb-Pb collisions,
suggesting a significant energy loss of the charm quark inside the hot and dense medium.
The central values of the $\RAA$ of $\Dsplus$ mesons are larger than those of non-strange D 
mesons. This would support the hypothesis that a fraction of the charm quarks hadronises via recombination 
with lighter quarks from the medium.
However, the current uncertainties need to be reduced to draw firmer conclusions.
For the first time, the anisotropies in the $\Dsplus$ azimuthal distribution were studied in semi-central Pb-Pb collisions.
In particular, the elliptic flow $v_2$ of $\Dsplus$ mesons was measured, showing positive values in the 
$\pt$ interval $2 <\pt <8 \, \Gevc$.
A positive $v_2$ indicates that charm quarks take part in the collective motion of the QGP.\\

Finally, a first measurement of the relative production yields of $\Dsplus$ and $\Dplus$ mesons as a function of 
multiplicity in p-Pb collisions $\sNN=5.02$ TeV was performed. This measurement is particularly interesting
after the observation of an increasing yield of strange particles relative to pions with increasing particle multiplicity 
in pp and p-Pb collisions, reaching values compatible with those observed in Pb-Pb collisions at similar multiplicities. 
Even though an intriguing trend of the $\Dsplus/\Dplus$ yield ratio is observed from pp to Pb-Pb 
collisions at low $\pt$, the large statistical and systematic uncertainties do not allow to conclude 
on the possible role of hadronisation via recombination in the smaller systems created in pp and p-Pb collisions. 
The larger data samples that will be collected during LHC Run 3 will allow to assess about
the $\Ds$ production as a function of multiplicity and more in general about charm quark in-medium hadronisation.


