\chapter{The ALICE experiment at the LHC}

\section{The Large Hadron Collider}
The Large Hadron Collider (LHC), with a circumference of 27 km, 
is the world's largest and most powerful particle collider, built by the 
European Organisation for Nuclear Research (CERN) from 1998 to 2008. 
It is placed in the tunnel of the previous Large Electron Positron collider, at a depth between 50 and 175 m underground.
The LHC accelerator chain is shown in Fig.~\ref{fig:imageLHC}. The first stage 
of the acceleration takes place on the Linac2, a linear accelerator with an 
output proton energy of 50 MeV. The proton-booster synchrotron (PSB) increases 
the energy to 1.4 GeV, injecting into the proton synchrotron (PS). This accelerates the
 beam to 26 GeV and injects into the super proton synchrotron (SPS), out of which 
 450 GeV protons are eventually injected into the LHC for the start of the ramp up to 
 the energy of 7 TeV. In the nominal configuration, LHC accelerates protons grouped in 2808 bunches per beam, each bunch 
 containing up to $1.15^{11}$ protons. The beam is bent along the circular LHC 
 path by 1232 superconducting dipoles and controlled and focused by another 
 600 smaller magnets. The design and construction of the dipoles was the most 
 technologically challenging part of the accelerator. To achieve the required bending power, 
 the dipole field should be on average B $\sim$ 8.3 Tesla. The coils are made of 
 NiTi superconducting cable, kept at T = 1.9 K by superfluid liquid He. They are 
 15 m long, weigh 35 tonnes and store in their magnetic field 7 MJ of energy, for a total of 10 GJ in the full ring.\\
The nominal istantaneus luminosity for pp collisions is of $10^{34} {\rm s}^{-1} {\rm cm}^{-2}$ , while for 
Pb-Pb collisions it is about $10^{27} {\rm s}^{-1} {\rm cm}^{-2}$. Since September 2008, pp
 beams circulated in the LHC ring at $\s=900, 2.36, 2.76, 7, 8$ and 13 TeV, 
 Pb-Pb beams at $\sNN=2.76$ and 5.02 TeV and p-Pb beams at $\sNN=5.02$ TeV.

\begin{figure}[!t]
\centering
\includegraphics[width=12cm]{FigCap3/CERN_Accl_Complex.png}
\caption{Layout of the full CERN accelerator complex, including all elements of the LHC injector chain.}
\label{fig:imageLHC}
\end{figure} 
\section{The ALICE experiment}
ALICE (A Large Ion Collider Experiment)~\cite{Abelev:2014ffa} is an experiment 
at the Large Hadron Collider optimized for the study of QCD matter
created in high-energy collisions between lead nuclei.
The aim of ALICE is the study of the behaviour of matter at high densities and temperatures 
and at near zero baryo-chemical potential. The detector consists of three main components: 
\begin{itemize}
\item \textbf{the central barrel}, contained in the large magnet with a weak solenoidal 
field (0.5 T) and composed of detectors devoted to the study of hadronic signals 
and di-electrons and covering the pseudo-rapidity range -0.9$< \eta <$0.9 over the 
full azimuth. They are the Inner Tracking System (ITS), optimised for vertex 
reconstruction and tracking; a cylindrical Time Projection Chamber (TPC),
 surrounding the ITS, that is the most important detector for tracking and provides particle identification; a 
 Transition Radiation Detector (TRD), designed for electron identification; 
 a Time Of Flight (TOF) detector, that provides pion, proton and kaon 
 identification; a Photon Spectrometer (PHOS); an Electromagnetic 
 Calorimeter (EMCal), a High-Momentum Particle IDentification (HMPID) 
 and the ALICE Cosmic Ray Detector (ACORDE);
\item \textbf{the forward muon spectrometer}, mainly dedicated to the study of 
the muon pairs from the decay of quarkonia, covering the pseudo-rapidity range $2.5< \eta < 4.0$;
\item \textbf{the forward detectors}, which include the Photon Multiplicity Detector 
(PMD) and the silicon Forward Multiplicity Detector (FMD), dedicated to the 
measurement of photons and charged particles around $|\eta| \sim 3$, respectively; 
the quartz Cherenkov detector T0 delivers the time and the longitudinal
 position of the interaction; the plastic scintillator detector V0 measures 
 charged particles at -3.7 $< \eta <$ -1.7 and 2.8 $< \eta <$ 5.1, and is mainly used
  for triggering and for the determination of centrality and event plane angle in 
  Pb-Pb collisions; the Zero Degree Calorimeter (ZDC) also used for determination of centrality.
\end{itemize}
The layout of ALICE set-up is shown in Fig.~\ref{fig:imageALICE}. 
\begin{figure}[!t]
\centering
\includegraphics[width=12cm]{FigCap3/alice.png}
\caption{Picture of the ALICE experiment detectors.}
\label{fig:imageALICE}
\end{figure}
ALICE apparatus has overall dimensions of 16 x 16 x 26 $m^3$ and a total 
weight of $\sim$10000 t. The maximum pp interaction rate at which all ALICE 
detectors can be safely operated is around 700 kHz. Typical luminosity values 
for the ALICE pp data taking range from \textit{L} $\sim 10^{29}{\rm s}^{-1}{\rm cm}^{-2}$ 
to \textit{L} $\sim 10^{31}{\rm s}^{-1}{\rm cm}^{-2}$.
The major challenge that the detector project had faced was the large number of 
particles created in Pb-Pb collisions. The design of the experiment was based 
on the highest value of particle production, 8000 charged particles per unit of 
rapidity, at mid-rapidity. For this reason, ALICE detectors was design to have a 
high granularity, a low transverse momentum threshold $\pt^{\rm min} \sim 0.15 $ 
$\Gevc$ and good particle identification capabilities up to 20 $\Gevc$.
In the following sections, the detectors of ALICE used in the analysis presented in this thesis and their 
performance will be described. For information on the other detectors more details can be found in~\cite{Aamodt:2008zz}.

\subsection{Magnet}
The magnet used in ALICE was constructed for the L3 experiment at LEP and 
it produces a relatively weak solenoidal magnetic field (B $<$ 0.5 T). In the choice 
of the magnetic field two aspects must be considered: the magnet has to be intense 
enough to bend the particle trajectory and allow a good track-momentum resolution, 
but still to permit the reconstruction of low 
momentum particles. The lower momentum that allows reconstruction of the track 
with the ALICE magnet is given by $p_{\rm cutoff}=$ 0.3$\,$B$\,$R $\sim 0.2\,\Gevc$, 
where B is the magnetic field in Tesla and R is the minimum radius for a particle to 
traverse the entire TPC, whose external radius is R$_{\rm ext}=$ 2.5 m.

\begin{figure}[!t]
\centering
\includegraphics[width=8cm]{FigCap3/figures_its-rf-2.png}
\caption{View of the six silicon layers of the Inner Tracking System}
\label{fig:image2}
\end{figure}

\subsection{Inner Tracking System (ITS)}
The Inner Tracking System of ALICE is one of the central detectors used for track 
reconstruction, primary and secondary vertex finding and Particle Identification (PID).
It is composed by six cylindrical layers of silicon detectors placed 
coaxially around the beam pipe (Fig.~\ref{fig:image2}). The layers are located 
at radii between 39 mm and 430 mm and cover the pseudo-rapidity range 
$|\eta|<0.9$. The two innermost layers are made of Silicon Pixel Detectors (SPD), 
their pseudo-rapidity coverage is $|\eta|<1.95$. The two middle layers are made of Silicon Drift Detectors (SDD) 
and two outer layers of Silicon micro-Strip Detectors (SSD).
Its basic functions are~\cite{ITS-TDR}:
\begin{itemize}
\item determination of primary and secondary vertices for charm, beauty and hyperon decay reconstruction,
\item particle identification and tracking of low-momentum particles,
\item improvement of the momentum and angle measurements of the TPC.
\end{itemize}
Because of high track-multiplicity values in Pb-Pb collisions,
 track finding is one the most challenging tasks in ALICE, and it is done by both the 
 ITS and the Time Projection Chamber (TPC). All the ITS detectors were carefully optimised to minimise their radiation 
 length, achieving 1.1\% per layer, the lowest value among all the current LHC experiments~\cite{ITS-TDR}. 
The resolution of the track impact parameter is determined by the spatial resolution of 
the ITS detectors. The ITS detectors have a spatial resolution of a few tens of 
$\mu$m in the $r\phi$ plane, with the best precision (12  $\mu$m) for the 
innermost detectors. All ITS detectors have a resolution in the plane perpendicular 
to the beam axis one order of magnitude better than that of the TPC, which in turn provides many more points.\\
 
 \begin{figure}[!h]
\centering
\includegraphics[width=.65\textwidth]{FigCap3/ITSpid.png}
\caption{Distribution of the energy-loss signal in the ITS as a function of momentum. The lines show the parameterisations of the expected mean energy loss.}
\label{fig:imagePIDITS}
\end{figure}

 The four outer layers provide analogue read-out and for this reason they 
 can be used for particle identification (PID) via d$E$/d$x$ measurements for 
 low-$\pt$ particles ($\pt \lesssim 0.7 \, \Gevc$) and in particular at $\pt\lesssim 0.5 \, \Gevc$
 where the ITS is used for standalone tracking. The measured cluster charge is normalised to the path length, 
 which is calculated from the reconstructed track parameters to obtain a d$E$/d$x$ 
 value for each layer. The d$E$/d$x$ values are calculated with a truncated mean, to assure that
 the d$E$/d$x$ peak shape is Gaussian. To this purpose, the d$E$/d$x$ is calculated as an 
 (weighted) average of the two lowest values out of four or three available points. 
 An example of the distribution of the measured truncated mean values of energy loss 
 per path unit as a function of track momentum in the ITS is shown 
 in Fig.~\ref{fig:imagePIDITS} (left). 
 \begin{figure}[!h]
\centering
\includegraphics[width=.6\textwidth]{FigCap3/TPC.png}
\caption{A view of the ALICE Time-Projection Chamber detector.}
\label{fig:imageTPC}
\end{figure}

\subsection{Time Projection Chamber (TPC)}
The TPC is the only device which can provide good 
performance up to 8000 charged particles per unit of rapidity~\cite{Dellacasa:451098}. 
It has a cylindrical shape with inner and outer radius of 80 and 250 cm, 
respectively, and an overall length in the beam direction of 500 cm. The minimum 
possible inner radius of the TPC is given by the maximum acceptable hit density. 
The outer radius is determined by the minimum length required for a d$E$/d$x$ resolution 
better than 10\%. At smaller radii (and larger track densities), tracking is taken over 
by the ITS. The TPC covers an acceptance of $|\eta|<0.9$. The TPC is a chamber full of 
high-purity gas to transport primary charges over long distances (2.5 m) towards the read-out end-plates. 
It is composed by a central high-voltage (HV) electrode which divides the 
gas volume into two symmetric drift regions, and two opposite axial potential degraders 
create a highly uniform electrostatic field of up to 400 V/cm (Fig.~\ref{fig:imageTPC}).
Charged particles traversing the gas form a ionisation trace that will move at constant 
velocity towards one of the two end-plates. The density of ionisation depends on the velocity and mass of the particle.
Once on the end-plate, readout chambers allow to amplify and register the signals 
of particle tracks. The end-plates are segmented into 18 trapezoidal sectors 
and equipped with multi-wire proportional chambers covering an overall active area of 32.5 $m^2$.
{\bf The mixture of gas is composed by 90\% Ne, 10\% ${\rm CO}_2$; more recently a 5\% 
N$_2$ was added in order to improve drift velocity. This gas mixture needs a high 
drift field (400 V/cm) to secure an acceptable drift time (92$\mu$s for the last mixture).} 
It is very important to provide high stability and uniformity in this mixture as it influences 
the precision in track reconstruction and energy-loss measurements. 
 \begin{figure}[!h]
\centering
\includegraphics[width=.65\textwidth]{FigCap3/TPCpid.png}
\caption{Distribution of the energy-loss signal in the TPC as a function of momentum. The lines show the parameterisations of the expected mean energy loss.}
\label{fig:imagePIDTPC}
\end{figure}
The TPC is the main detector for track reconstruction (see Sec.~\ref{sec:tracking}) and provides a 
tracking efficiency larger than 90\%. It is used for measurements of the charged-particle 
momenta, with a resolution better than 2.5\% for electrons with momentum 
of about 4 $\Gevc$, and for particle identification (d$E$/d$x$). It allows
two-track separation in the region $\pt <10$ $\Gevc$.\\


Particle Identification is performed over a wide momentum range. It is made by 
simultaneously measuring the specific energy loss, the charge and the momentum of the 
particles traversing the detector gas. The energy loss, described by the Bethe Block formula,
is parameterised by a function proposed by ALEPH Collaboration~\cite{Rolandi:2008qla}:
\begin{equation}
f(\beta_\gamma) = \frac{P_1}{\beta^{P_4}} \Big ( P_2 - \beta^{P_4} - {\rm ln}\Big (   P_3 + \frac{1}{(\beta_\gamma)^{P_5}} \Big )   \Big ),
\end{equation}
where $\beta$ is the particle velocity, $\gamma$ is the Lorentz factor, and $P_{1-5}$ are fit parameters. 
The measured d$E$/d$x$ as a function of the track momentum measured in TPC is
shown in Fig.~\ref{fig:imagePIDTPC}. The lines correspond to the parametrisation. 
It is evident a clear separation between the different particle species
at $\pt \lesssim 1 \, \Gevc$. At higher $\pt$ the separation of the species is still feasible 
on a statistical basis via multi-Gaussian fits. The PID resolution is about 5.2\% in 
pp collisions and 6.5\% in the 0-5\% most central Pb-Pb collisions. Thanks to the d$E$/d$x$ resolution, 
particle ratios can be measured at high $\pt$, up to 20 $\Gevc$.
 \begin{figure}[!h]
\centering
\includegraphics[width=.6\textwidth]{FigCap3/TOFCylinder.png}
\caption{A view of the ALICE Time-Of-Flight detector.}
\label{fig:imageTOF}
\end{figure}
\subsection{Time-Of-Flight (TOF)}
The TOF detector is a large area array of Multigap Resistive Plate Chambers (MRPC), 
positioned at 370-399 cm from the beam axis. It has a cylindrical 
shape, covering polar angles between 45 and 135 degrees over the full azimuth. 
The TOF has a modular structure with 18 sectors in $\varphi$ (Fig.~\ref{fig:imageTOF}); each 
of these sectors is divided into 5 modules along the beam direction. 
This detector is entirely devoted to particle identification.
The TOF resolution is 80 ps for pions with a momentum around 1 $\Gevc$, in 0-70\% Pb-Pb collisions.
This values includes the detector resolution, the contribution from electronics, the uncertainty
on the start time of the event and the tracking and momentum resolutions. 
It provides PID in the intermediate momentum range up to 2.5 $\Gevc$ for pions and kaons and 4 $\Gevc$ for 
protons. The ionisation produced by any particle crossing the MRPCs starts a gas avalanche 
which generates the observed signal. The TOF identifies the particle species via a measure of the time of flight 
inside the chambers. Considering the following relation:
\[
m = p \sqrt{\frac{t^2_{\rm TOF}}{L^2}-1}
\]
where $m$ is the mass of the particle, $p$ the momentum, $t_{\rm TOF}$ the time-of-flight and 
$L$  the track length, it is simple to show that the $\delta m/m$ resolution, at relatively high
 momenta, it is influenced much more by the errors on the time-of-flight and track length 
 measurements than by error on the momentum determination.
 The start time for the TOF detector is generally provided by T0 detector, which is composed by two arrays of
Cherenkov counters, T0A and T0C, located at apposite sides of the interaction point at 
$-3.28 < \eta < - 2.97$ and $4.61 < \eta < 4.92$. It has a time resolution of 20-25 ps
 in Pb-Pb collisions and $\sim 40$ ps in pp collisions.
Figure \ref{fig:TOFpid} shows the distribution of the measured velocity $\beta$ as a function of momentum (measured by TPC). 
The background is due to tracks that are incorrectly matched to TOF hits in high-multiplicity Pb-Pb collisions.
\begin{figure}[!h]
\centering
\includegraphics[width=.6\textwidth]{FigCap3/TOFpid.png}
\caption{Distribution of $\beta$ velocity measured by the TOF detector as a function of momentum for particles reaching the
TOF in Pb-Pb interactions.}
\label{fig:TOFpid}
\end{figure}

\iffalse
\subsection{Transition Radiation Detector (TRD)}
The Transition Radiation Detector \cite{TRD-TDR} is placed between the
 Time Projection Chamber and the Time-Of-Flight detectors. It is the main 
 detector in the central barrel revealing high momentum electrons, with $\pt >$1 $\Gevc$, 
 i.e. the region where it is no more possible to use only the information from 
 energy loss of TPC for particle identification. Since the TRD is a fast tracker, 
 it is possible to use it as an efficient trigger for high-momentum electrons. 
 As the other detectors of the barrel, the TRD covers an acceptance of $|\eta|<$0.9.
The transition radiation is produced when a highly relativistic charged particle 
traverses the boundary between two media of different dielectric constants. 
The probability of transition radiation increases with the relativistic gamma factor 
and this provides an excellent way to discriminate between electrons and pions for
 momenta of a few $\Gevc$ and higher. The probability of emission traversing 
 a single boundary is small, so multiple boundaries are necessary to obtain a reasonable efficiency.
It consists of 540 detector modules, each one composed by a radiator of 4.8 cm 
thickness, a multi-wire proportional readout chamber, and the front-end electronics
 for this chamber. The gas mixture in the readout chamber is Xe/CO\textsubscript{2} 
 (85\%/15\%). Each readout chamber consists of a drift region of 3.0 cm separated 
 by cathode wires from an amplification region of 0.7 cm. The drift time is 2.0 $\mu$s.



\subsection{High Momentum Particle Identification (HMPID)}
The High Momentum Particle Identification system plays a role in the particle
 identification of heavy-ion collisions in ALICE \cite{HMPID-TDR}. It allows to extend
  the PID capabilities beyond the momentum range allowed by energy loss measurements 
  (ITS and TPC, $p \sim$ 600 MeV/c) and by the TOF ($p \sim$ 1.2-1.4 $\Gevc$). 
  The HMPID detector has been designed to extend the useful range for identification of 
  $\pi$/K up to 3 $\Gevc$ and of p/K up to 5 $\Gevc$. 


This detector consists of seven modules of about 1.5$\times$1.5 m\textsuperscript{2} 
each of Ring Imaging Cherenkov (RICH) counters. Cherenkov radiation is a wave resulting
 from charged particles when traversing a material faster than the velocity of light in that 
 material. The radiation propagates with a characteristics angle, that depends on the particle 
 velocity. The radiator is a thick layer of $C_6F_{14}$ (perfluorohexane) with an index of 
 refraction $n$=1.2989 at $\lambda$=175 nm, corresponding to a momentum threshold 
 of $p_{th}$=1.21$\times$m, where m is the particle mass. Then, a photon detector made 
 of CsI is used to detect the ring-shaped image of the Cherenkov radiation. From a 
 measurement of the Cherenkov angle and thus the particle velocity, the mass of the charged particle can be estimated.

\subsection{Photon Spectrometer (PHOS)}
PHOS (PHOton Spectrometer) is an electromagnetic calorimeter of high granularity 
made of lead tungstate crystals \cite{PHOS-TDR}. It is positioned at the bottom of the 
ALICE set-up and covers about a quarter of unit in pseudo-rapidity, $-0.12<\eta<0.12$ and $100\degree$ 
in azimuthal angle. The main aim of the PHOS detector is the determination of the 
thermal and dynamical properties of the initial phase of the collision. This is done 
by measuring photons emerging directly from the collisions, that have to be 
separated by photons coming from particle decays, and also neutral mesons like 
$\pi^0$ and $\eta$ through their decays in two photons.

\subsection{Electromagnetic Calorimeter (EMCal)}
The Electromagnetic Calorimeter EMCal enhances ALICE's capabilities for jet 
quenching measurements \cite{EMCal-TDR}.
The detector contains several modules each consisting of sampling calorimeters,
 in fact the EMCal extends the ALICE $\pt$ capabilities for jets, direct photons and 
 electrons from heavy-flavour decays. It is made of alternating layers of 1.44 mm Pb 
 and 1.76 mm polystyrene, which is the scintillating material. The EMCal covers the
  pseudo-rapidity range $-0.7 < \eta < 0.7$.

\subsection{Forward Muon Spectrometer (FMS)}
The Forward Muon Spectrometer is used to evaluate the production of $\mu^+\mu^-$
 pairs coming from the decay of heavy quarkonium states, like the charmonium states 
 (J/$\Psi$ and $\Psi'$) and the bottomonium states ($\Upsilon, \Upsilon'$ and $ \Upsilon"$).
  In presence of a deconfined medium, quarkonium states are dissociated because 
  of colour screening and this leads to a suppression of their production rates.

The spectrometer is located around the beam pipe and its acceptance covers
 the pseudo-rapidity interval 2.5$\leq \eta \leq$4 \cite{FMS-TDR}. A resolution 
 of 70 MeV/$c^2$ in the 3 GeV/$c^2$ region is needed to separate J/$\Psi$ and 
 $\Psi'$ peaks and of 100 MeV/$c^2$ in the 10 GeV/$c^2$ region to distinguish $\Upsilon, \Upsilon' $and $\Upsilon"$.
This detector is composed by a front absorber (Fig.~\ref{fig:muon} left) which 
suppresses all the particles except the muons. It is made by carbon and concrete
 in order to limit the energy loss and multiple scattering of the muons. Then, a tracking 
 chamber allows to obtain muon tracks, thanks to multi-wire proportional chambers, 
 with a spatial resolution better than 100 $\mu$m, and a dipole magnet outside the L3
  magnet which bends the charged particles tracks. The trigger on di-muon signals is 
  provided by four layers of RPC operating in streamer mode that are located behind an iron absorber.
\begin{figure}[!t]
\centering
\includegraphics[width=.49\textwidth]{FigCap3/dimuon1.jpeg}
\includegraphics[width=.49\textwidth]{FigCap3/dimuon2.png}
\caption{(Left) Basic principle of the Dimuon spectrometer: an absorber to filter the background, a set of tracking chambers before, inside and after the magnet and a set of trigger chambers. (Right) Mass spectrum of principal resonances of quarkonium states.}
\label{fig:muon}
\end{figure}
\fi

\subsection{V0 Detector} 
\label{sec:V0}
The V0 detector is made of two arrays of scintillator counters, V0A and V0C, positioned on both sides of 
the ALICE interaction point. The V0A detector is located at 340 cm from the vertex, on the side opposite to the muon spectrometer,
whereas V0C is fixed to the front face of the hadronic absorber, 90 cm from the vertex. They cover
the pseudo-rapidity ranges $2.8 < \eta < 5.1$ (V0A) and $-3.7 < \eta < -1.7$ (V0C) and are segmented
into 32 individual counters each distributed in four rings. The V0 provides a minimum bias trigger for the central barrel detectors and
  can be used to estimate the centrality of the collision by summing up the energy deposited in the two V0.
This detector provides minimum-bias triggers for the central barrel
detectors in pp and A-A collisions. The V0 is also an indicator of the centrality of the collision via the multiplicity
recorded in the event. Furthermore, it is used for beam-gas background rejection (see Sec.~\ref{sec:BkgRejection}) 
and for luminosity measurements. 


\subsection{The Zero Degree Calorimeter (ZDC) Detector} 
\label{sec:ZDC}
The ZDC provides a measure of the energy carried in the forward direction (at 0\degree relative
to the beam direction) by non-interacting (spectator) nucleons. This quantity is
useful to estimate the centrality of the event. Typically the beams are deflected by means 
of two separation dipoles at a certain distance from the interaction point (IP). These magnets will also deflect the spectator
 protons, separating them from the spectator neutrons, which basically fly away at 0. 
Two sets of ZDCs are placed 115 m away from the 
 interaction point on both sides of the interaction point. Each set is constituted by one 
 calorimeter for neutron detection (ZN), positioned between the two 
 beams to intercept the spectator neutrons, and one for proton detection (ZP), placed externally 
 to the outgoing beam, to collect the spectator protons. 
  The hadronic ZDCs are quartz fibres sampling calorimeters. The shower generated by incident
particles in a dense absorber produces Cherenkov radiation in quartz fibres (the active material) interspersed in the absorber.
The two sets of ZDCs are complemented by the information of two small 
electromagnetic calorimeters (ZEM), located at 7.5 m from the interaction
 point, to detect the  energy of particles emitted at forward rapidity (mainly photons generated from $\pi^0$ decays).




\section{The ALICE Trigger System and Data Aquisition}
\label{sec:trigger}
ALICE has a two-layer trigger architecture~\cite{Fabjan:684651}. The low-level trigger is a 
hardware trigger called Central Trigger Processor (CTP). The High-Level trigger (HLT)
 is implemented as a pure software trigger. 
The ALICE Central Trigger Processor (CTP) is designed to combine and synchronise 
information from all the triggering detectors in ALICE, and to send the correct sequences
 of trigger signals to all detectors in order to make them read-out properly. Since the ALICE 
 experiment has to do with pp and Pb-Pb collisions, the trigger system was optimised 
 for both these types of collisions. The HLT allows the implementation of sophisticated 
 logic for the triggering. While the CTP governs the readout of the sub-detectors, the 
 HLT receives a copy of the data read out from the sub-detectors and processes it.

\subsection{The Central Trigger Processor (CTP)}
\label{sec:CTP}
The hardware trigger combines informations from sub-detectors to decide whether or 
not to write an event on disk. The CTP evaluates trigger inputs from the trigger detectors every machine clock cycle ($\sim25$ ns). 

The trigger inputs are divided into three different levels, 
because of the dimension of the detector:
\begin{itemize}
\item L0 level: the Level 0 trigger decision (L0) is made $\sim 0.9\, \m$s after the 
collision using V0, T0, EMCal, PHOS, and MTR. The trigger requirement can be 
simply the input of one detector or a logical condition based on the trigger 
inputs of different trigger detectors.
\item L1 level: the events accepted at L0 are further evaluated by the Level 1 (L1) 
trigger algorithm in the CTP. The L1 trigger decision is made $\sim 6.5\, \mu$s after L0. 
The latency is caused by the computation time (TRD and EMCal) and propagation times 
(ZDC, 113 m from IP2). The L0 and L1 decisions, delivered to the detectors with a 
latency of about 300 ns, trigger the buffering of the event data in the detector front-end electronics. 
\item L2 level: the Level 2 (L2) decision, taken after about 100 $\mu$s corresponding to the 
drift time of the TPC, triggers the sending of the event data to DAQ and, in parallel, to the High Level Trigger system (HLT). 
The L2 latency is used to avoid pile-up effects, due to both the high luminosity and the slowness of some 
detectors. When the L2 requirements are fulfilled, the event can be sent to the Data Acquisition 
System (DAQ). The DAQ manages the data flow from the sub-detector electronics to the archiving on tape.
\end{itemize}
ALICE operates with minimum-bias (MB) triggers, mainly based on V0 and SPD, 
and with rare triggers that are optimised to select particular classes of events such as 
events containing jets or muons or high-multiplicity events. By definition MB triggers 
have the highest rate of inputs signals, while the rare triggers have much lower rate. 
In general, several types of triggers during the data taking at the same time keep busy a 
very large fraction of the total data acquisition bandwidth. To prevent losing precious events due to the fact that no 
space is available on the temporary memory, the trigger system follows an event prioritisation scheme. 
In the case that the utilisation of the temporary storage is above a certain 
value, only rare triggers are accepted. This scheme significantly increases the acceptance of rare events.
Information about the LHC bunch filling scheme is used by CTP to suppress the background. 
The information about the bunch crossing, regarding the arrival of bunches from both A-side and C-side,
or one of them, or neither, is provided to the CTP by the bunch crossing mask at a resolution of 25 ns. 
The beam-gas interaction background, studied by triggering on bunches without a collision partner, is 
subtracted from the physics data taken with the requirement of the presence of both bunches.

\subsection{The Data AcQuisition System (DAQ)}
\label{sec:DAQ}
The DAQ manages the data flow from the sub-detector electronics to the archiving
 on tape. Once the CTP decides to register a specific event, raw data are sent to 
 the Local Data Concentrators (LDCs) via the optical Detector Data Links (DDLs). 
 LDCs are sets of computers that perform sub-event reconstruction. In this step of 
 the acquisition, raw data are processed. The size of a single central event processed 
 by LDCs can be $\approx$70 MB. In order to optimise the usage of the recording bandwidth available, 
 an additional event selection and compression is done by the High-Level Trigger (HLT).
The events processed by LDCs are then transferred to a second layer of computers, the Global Data Collectors 
 (GDCs) which perform the event building.
  
 \subsection{The High Level Trigger (HLT)}
\label{sec:HLT}
ALICE software trigger, called HLT, is a farm of multiprocessor computers. 
It is composed by about 1000 PCs processing the data in parallel allowing an 
online analysis of the events. The HLT receives a copy of the raw data 
and performs per-detector reconstructions. Then, the trigger decision is based on 
the global reconstructed event. A trigger decision is derived from much more complete 
information than that available for the hardware trigger. Therefore, it allows for more 
sophisticated triggers. Examples include triggers on high-energy jets or on muon pairs. 
Data rate reduction is achieved by reducing the event rate by selecting interesting events 
(software trigger) and by reducing the event size by selecting sub-events (e.g. pile-up 
removal in pp interactions) and by advanced data compression. 
The trigger decision, partial readout information, compressed data, and the reconstruction output is sent to LDCs and 
subsequently processed by the DAQ. 

\section{Background rejection}
\label{sec:BkgRejection}
The operation of detectors at LHC can be affected by machine-induced
background (MIB), originating from the beam interactions with the matter in the machine.
This background scales with beam intensity and mainly depends on the 
residual gas pressure in the beam pipe and the cleaning efficiency of collimator systems.
The background rejection is done by exploiting the arrival time of the signal in the two V0 modules.
The background caused by one of the LHC beams produces an early
signal on one of the two V0 (depending on the side from where the beam arrives) 
compared to the collision time at the nominal interaction point.
The difference between the expected beam and background signals 
is about 22.6 ns in the A side and 6 ns in the C side. As shown in the left panel of Fig.~\ref{fig:V0sumdiff}, 
background events accumulate mainly in two peaks at (-14.3 ns, -8.3 ns)
and at (14.3 ns, 8.3 ns) in the V0 time sum-difference plane, 
well separated from the main (collision) peak at (8.3 ns, 14.3 ns). 
\begin{figure}[!h]
\centering
\includegraphics[width=.49\textwidth]{FigCap3//V0sumdiff.png}
\includegraphics[width=.49\textwidth]{FigCap3//SPDcltsVsTcklts.png}
\caption{Left: correlation between the sum and difference of signal times in 
V0A and V0C. Right: correlation between reconstructed SPD clusters and tracklets. }
\label{fig:V0sumdiff}
\end{figure}
The V0 time information is 
used to set the trigger conditions on collision or background events. 
The collected events are further selected offline to remove any residual
contamination from MIB and satellite collisions.
The V0 trigger logic is validated using a V0 arrival time 
computed offline using a weighted average of all detector elements.
In pp physics, an additional information on the correlation
between number of hits and track segments (tracklets) in SPD detector in used.
In fact, background particles usually cross the pixel layers in a 
direction parallel to the beam axis, producing hits that are not associated 
to any tracklets in the reconstruction (right panel of Fig.~\ref{fig:V0sumdiff}).
The dashed green line represents the cut used in the offline selection: 
events lying in the region above the line are tagged as background and rejected.
This cut rejects a negligible number of events beyond those already rejected by the V0. 
The fraction of background events that survives the above cuts was determined
by a control data taking with only one of the beams crossing the ALICE 
interaction point and was found to be about 0.3\% in the pp data taking 
during the 2010 run, but it strongly depends 
on the running conditions and on the specific trigger configuration under study. 
In Pb-Pb collisions the fraction was found to be smaller than 0.02\%.
Collisions of main bunches and satellite bunches located at short 
distance from the main bunch are also a source of background. 
Satellite events are rejected using the correlation between the sum and the difference 
of times measured in the ZDC, as shown in Fig.~\ref{fig:ZNAselection} for Pb-Pb collisions.
The large cluster in the middle corresponds to collisions between ions 
in the main bunches, while the small clusters along the diagonals (spaced by 
2.5 ns in time difference) correspond to collisions in which one of the ions belongs
to satellite bunches.
\begin{figure}[!h]
\centering
\includegraphics[width=.49\textwidth]{FigCap3/ZNAselection.png}
\caption{Correlation between the sum and the difference of times recorded by the neutron ZDCs on either side (ZNA and ZNC) in Pb?Pb collisions. }
\label{fig:ZNAselection}
\end{figure}


\section{Track and vertex reconstruction}
\label{sec:tracking}
The track reconstruction and vertex finding algorithm are performed offline using the central
barrel detectors (ITS and TPC). The main steps are described below.
\begin{itemize}
\item {\bf Clusterisation step:} the detector data are converted into ''clusters'' characterised 
by positions, signal amplitudes, signal times, etc., and their associated errors. The clusterisation 
is performed separately for each detector. 
\item {\bf Preliminary (SPD) interaction vertex finding:} the clusters found in the SPD detector are used to determine the tracklets,
i.e. segments of tracks build by associating pairs of reconstructed points close in azimuthal angle ($\Delta \varphi <$ 0.01 rad) and 
pseudo-rapidity in the two SPD layers.
The preliminary vertex is defined as a space point where the maximum number of tracklets converge.
In pp collisions, where interaction pileup is expected, the algorithm is repeated several times, discarding at 
each iteration those clusters which contributed to already-found vertices. When a primary vertex is not found 
(particularly in low-multiplicity events), the algorithm performs a one-dimensional 
search in the $z$-distribution of the points of closest approach of tracklets to the nominal beam axis.
\item {\bf TPC track finding:} track finding and fitting is performed in three stages, following an inward-outward-inward scheme.
The first inward stage starts with track finding in the TPC. The TPC can produce a maximum of 159 clusters per track
(corresponding to the 159 tangential pad rows). The search for track clusters starts at large TPC radius.
Track seeds are built with two TPC clusters and the preliminary vertex point, then with three clusters 
and without the vertex constraint. The track seeds are propagated inward and, at each step, 
the nearest cluster is added. A special algorithm is used to reject tracks with a fraction of common clusters
larger than a certain limit (between 20\% and 50\%). Only tracks that have at least 20 clusters (out of the maximum 159)
and that miss no more than 50\% of the clusters expected for a given track position are accepted. 
The contamination by tracks with more than 10\% wrongly associated clusters does not exceed 3\%
even in most central Pb-Pb collisions.
The efficiency of tracking in TPC is defined as the ratio between the reconstructed tracks 
and the generated primary particles in the simulation, and is shown in Fig.~\ref{fig:TPCtrackEffAndME} (left) as a function of 
the transverse momentum of the track. The tracking efficiency steeply drops below $\pt \sim 0.5 \, \Gevc$
due to the interaction of the particles with the detector material. The efficiency is almost 
independent of the occupancy in the detector. 
\begin{figure}[!h]
\centering
\includegraphics[width=.49\textwidth]{FigCap3/TPCtrackEff.png}
\includegraphics[width=.49\textwidth]{FigCap3/ITSTPCmatchEff.png}
\caption{Left: TPC track finding efficiency for primary particles in pp and Pb-Pb collisions (simulation)~\cite{Abelev:2014ffa}. Right: ITS-TPC matching efficiency versus $\pt$ for data and Monte Carlo for Pb-Pb collisions~\cite{Abelev:2014ffa}.}
\label{fig:TPCtrackEffAndME}
\end{figure}
\item{\bf Standalone ITS track finding:} the reconstructed tracks in TPC are propagated inwards the ITS.
Track seeds in the ITS are defined from the outermost ITS layer and then updated at each ITS layer by 
all clusters that satisfy a proximity cut. Each TPC track is hence associated to a tree of tracks in the ITS.
The track candidates in ITS are then selected with quality cuts (reduced $\chi^2$ and number of shared clusters with
other tracks). The track prolongation efficiency from TPC to ITS in Pb-Pb collisions is shown in Fig.~\ref{fig:TPCtrackEffAndME} (right),
as a function of $\pt$ of the tracks for different requests on ITS points. 
The data and Monte Carlo (MC) efficiencies are shown by solid and open symbols, respectively. 
Since the track efficiency in TPC allows to reconstruct tracks down to $\pt \sim 200$ MeV/$c$ for pions 
and $\pt \sim 400$ MeV/$c$ for protons, the ITS reconstruction is performed with those clusters 
that were not used in the ITS-TPC tracks. The used algorithm allows to track low-momentum particles 
down to $\pt \sim 80$ MeV/$c$.
\item{\bf Track outward propagation:} all tracks are extrapolated to their point of closest approach to the 
preliminary interaction vertex and the outward propagation starts. A Kalman filter is used to re-fit the tracks 
in the outward direction using the clusters found at the previous stage. 
The track length integral and the time of flight expected for various particle species are updated for 
subsequent particle identification with TOF. The tracks are matched to TOF clusters and to TRD tracklet
in each of the six TRD layers, if possible. The matching with outer detectors (EMCal, PHOS and HMPID)
is also attempted. However, only the TPC and the ITS are used to updated the measured track kinematics.
\begin{figure}[!h]
\centering
\includegraphics[width=.49\textwidth]{FigCap3/DCAxyResol.png}
\includegraphics[width=.49\textwidth]{FigCap3/ptResol.png}
\caption{Left: Resolution of the transverse distance to the primary vertex for charged ITS-TPC tracks. The contribution from the vertex resolution is not subtracted~\cite{Abelev:2014ffa}. Right: $\pt$ resolution for TPC standalone and ITS-TPC tracks with and without the constraint to the vertex~\cite{Abelev:2014ffa}. }
\label{fig:DCAxyResolAndPtResol}
\end{figure}
\item{\bf Track inward propagation:} the tracks are propagated inwards starting from the outer radius of the TPC. 
The tracks are refitted in TPC and ITS with the previously found clusters. The position, direction, inverse curvature of the track 
and its associated covariance matrix are determined. The resolution of the transverse distance to the 
primary vertex for charged-particle ITS-TPC tracks in pp, p-Pb and Pb-Pb collisions is shown in Fig.~\ref{fig:DCAxyResolAndPtResol} (left). 
The resolution improves in heavier systems thanks to large multiplicities that allow a more precisely 
vertex determination. The relative transverse momentum resolution of the tracks is related to the
resolution of the inverse-$\pt$ resolution via:
\begin{equation}
\label{eq:ptResol}
\frac{\sigma_{p_{\rm T}}}{\pt} = \pt \, \sigma_{1/p_{\rm T}}.
\end{equation}
The inverse-$\pt$ resolution for TPC standalone tracks and ITS-TPC combined tracks 
is shown in Fig.~\ref{fig:DCAxyResolAndPtResol} (right), for p-Pb collisions as a function of the inverse $\pt$.
In central Pb-Pb collisions, the $\pt$ resolution deteriorates by $\sim 10-15\%$ at high $\pt$. 
\item{\bf Final interaction vertex finding:} global ITS-TPC tracks are used to find the interaction vertex position 
with a higher precision than with SPD tracklets alone. The transverse resolution of the 
preliminary interaction vertices found with SPD and of the final ones with global tracks are shown in the left panel of Fig.~\ref{fig:VtxResol}. 
Both resolutions scale with the square root of the number of contributing tracks.
\begin{figure}[!h]
\centering
\includegraphics[width=.49\textwidth]{FigCap3/VtxReso.png}
\includegraphics[width=.49\textwidth]{FigCap3/vertexSec.pdf}
\caption{Left: transverse resolution of the final vertex distribution (solid points) and preliminary SPD interaction vertices (open points) are shown as a function of the track multiplicity~\cite{Abelev:2014ffa}. Right: resolution of the reconstructed secondary vertex along $\Dplus$ $\pt$ direction (black), 
orthogonal to $\Dplus$ $\pt$ direction (red), and along $z$-axis (blue). }
\label{fig:VtxResol}
\end{figure}
\end{itemize}

\section{Secondary vertex reconstruction}
For the products of heavy-flavour decays, the reconstruction of decay vertices 
is done during the analysis phase. Tracks are grouped in triplets (for the $\Ds$ case) 
following the charge ordering of the decay channel, defining objects called ``candidates''. 
The algorithm used is based on a straight 
line approximation of the tracks (which are helices): tracks are approximated as straight 
lines close to the primary vertex, by calculating the tangent line.
The algorithm finds the point of minimum distance between the three tracks by minimising the quantity:
\begin{equation}
D^2=d^2_1+d^2_2+d^2_3
\end{equation}
where $d_i$ ($i$=1,2,3) is the distance between the track $i$ and the vertex 
($x_0$, $y_0$, $z_0$), weighted with the errors on the track:
\begin{equation}
d^2_i=\left(\frac{x_i-x_0}{\sigma_{x_i}}\right)^2+
\left(\frac{y_i-y_0}{\sigma_{y_i}}\right)^2+\left(\frac{z_i-z_0}{\sigma_{z_i}}\right)^2.
\end{equation}
The resolution of the reconstructed secondary vertex is shown in Fig.~\ref{fig:VtxResol} (right) 
(right) as example for the $\Dplus$ meson. The 
resolution improves in the perpendicular plane to the $\Dplus$ $\pt$ direction.



\section{Event simulation and reconstruction}
Event generators create primary particles, depending on the physics one is interested in. 
Then, the physics processes at partonic level and informations such as type, momentum, 
charge, mother/daughter relationships... are stored in the kinematics tree. Different
 programmes are then used to simulate the transport of the particles through the
  detectors. During the transport, the energy deposition in the various detectors is stored 
  as ''hits''. The hits are subsequently converted into ''digits'', which represent 
  the real detector response and take into account instrumental effects, such as the noise 
  due to the front-end electronics. Finally, the digit are stored in a specific hardware format 
  for each detector as ''raw'' data, that corresponds to the data format coming from the 
  Data Acquisition System in a real data taking. From this point on the reconstruction starts, 
  without distinguishing between real or simulate data. 


\section{The ALICE Offline Software Framework}
In ALICE, on average pp and Pb-Pb events have a size of about 1.1MB and 13.75 MB respectively. 
For a standard running year, an order of $10^9$ pp events and $10^8$ Pb-Pb events are expected, for a total 
raw data volume of 2.5 PB. The processing and analysis of these data necessitate 
unprecedented amount of computing and storage resources. Grid computing provides 
the answer to these needs. Grid computing consists of a coordinated use of large sets 
of different, geographically distributed resources in order to allow high-performance 
computation. It is organised in different levels or Tiers. Data coming from LHC experiments 
are stored in the CERN computing centre, the Tier-0. Copies of the collected data are then 
replicated in large regional computing centres (Tier-1), which also contribute in the event 
reconstruction and Monte Carlo simulation. Tier-2 centres are computing centres located
 in different institutions which do not have large storage capabilities but provide a large 
 fraction of the computing resources for Monte Carlo simulation, data reconstruction and 
 data analysis. ALICE uses the ALICE Environment (AliEn) system~\cite{Saiz:2003wi} as a 
 user interface to connect to a Grid composed of ALICE specific services that are part 
 of the AliEn framework and basic services of the Grid middleware installed at the different sites.

\subsection{The AliRoot Framework}
The ALICE offline framework, AliRoot~\cite{Aamodt:2008zz} is based on Object-Oriented techniques
 for programming and, as a supporting framework, on the ROOT system~\cite{Brun:1997pa}, 
 complemented by the AliEn system which gives access to the computing Grid. The 
 AliRoot framework was developed as an extension of ROOT and is used for simulation, 
 alignment, calibration, reconstruction, visualisation and analysis of the experimental data.
\begin{figure}[h]
\centering
\includegraphics[width=10cm]{FigCap3/aliroot.png}
\caption{Schematic view of the AliRoot framework.}
\label{fig:aliroot}
\end{figure}
TheAliRoot framework is schematically shown in Fig.~\ref{fig:aliroot}. The STEER module 
provides steering, run management, interface classes, and base classes. The detectors are 
independent modules that contain the code for simulation and reconstruction while the 
analysis code is progressively added. Detector response simulation can be performed 
via different transport  packages like GEANT3~\cite{Brun:1082634} and FLUKA~\cite{Ferrari:898301} 
written in FORTRAN and GEANT4~\cite{Agostinelli:2002hh} written in C++. In these packages 
the detector material budget is simulated in detail, including support structures and 
the beam pipe. The transport code can hence simulate the decays of unstable 
particles and the trajectory of the daughter particles, the interactions of the 
particles with the detectors material and the production of secondary electrons.
The reconstruction results are stored in ESDs (Event Summary Data), from which 
the analysis tasks produce the AODs (Analysis Object Data) specialised in the physics 
objectives and used for the analysis (for example, specific AOD are produced for the 
reconstruction of open-charm meson with two or three body decays).

